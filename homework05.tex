\documentclass{article}
\usepackage{graphicx} 
\usepackage{theJackPack}

\title{Homework 5}
\author{Jack Brolin, Abhiram Nallamalli}
\date{July 2023}

\begin{document}

\maketitle

\begin{jacklist}
    \begin{framed} 
    \item [\textbf{P. 1}] Consider the linear programming problem: 
        \begin{align*}
            \text{min }  & x_{1}-x_{2} \\
            \text{s.t. } & 2 x_{1}+3 x_{2}-x_{3}+x_{4} \leq 0 \\
                         & 3 x_{1}+x_{2}+4 x_{3}-2 x_{4} \geq 3 \\
                         & -x_{1}-x_{2}+2 x_{3}+x_{4}=6 \\
                         & x_{1} \leq 0 \\
                         & x_{2}, x_{3} \geq 0 .
        \end{align*} 
        Write down the corresponding dual problem.
    \end{framed}
    Following the table from section 4.2 of the text, the corresponding dual problem is as follows:
    \begin{align*}
        \text{Max } & 3p_2 + 6p_3 \\
        \text{s.t. } & 2p_1 + 3p_2 - p_3 \geq 1\\
                     & p_1 + p_2 - p_3 \leq -1 \\
                     &-p_1 + 4p_2 + 2p_3 \leq 0 \\
                     &p_1 - 2p_2 - p_3 = 0 \\
                     &\qquad \qquad  p_1 \leq 0 \\
                     &\qquad \qquad  p_2 \geq 0 \\
                     & \qquad \qquad p_3 \text{ is free}
    \end{align*}
\newpage
    \begin{framed} 
    \item [\textbf{P. 2}] Consider a linear programming problem in standard form and assume that the rows of $A$ are 
        linearly independent. For each one of the following statements, provide either a proof or a counterexample.
        \begin{itemize}
            \item [a.] Let $x^{*}$ be a basic feasible solution. Suppose that for every basis corresponding to $x^{*}$, 
                the associated basic solution to the dual is infeasible. Then, the optimal cost must be 
                strictly less than $c^{\prime} x^{*}$.
            \item [b.] The dual of the auxiliary primal problem considered in Phase I of the simplex method is always feasible.
            \item [c.] Let $p_{i}$ be the dual variable associated with the $i^{th}$ equality constraint in the primal. 
                Eliminating the $i^{th}$ primal equality constraint is equivalent to introducing the additional constraint 
                $p_{i}=0$ in the dual problem.
            \item [d.]  If the unboundedness criterion is satisfied when we solve the primal with the simplex method, 
                then the dual problem is infeasible. 
        \end{itemize}
    \end{framed}
    \begin{itemize}
        \item [a.] True. We know that the optimal cost of the dual is less than or equal to the optimal most of the primal. If 
            $x^\star$ is a basic feasible solution of the primal but infeasible for the dial, the its cost me be greater than the cost
            of the dual. 
        \item [b.] True. Since the primal auxiliary problem is always feasible, the dual is feasible too.
        \item [c.] True. Constraints are of the form $\sum p_ia_i$, so setting $p_i = 0$ effectively eliminates it. 
        \item [d.] True. If the primal cost goes to $- \infty$ than the dual is infeasible. 
    \end{itemize}
\newpage
    \begin{framed} 
    \item [\textbf{P. 3}] Let $A$ be a symmetric square matrix. Consider the linear programming problem: 
        \begin{align*}
            \text{minimize } & c^\prime x \\
            \text { subject to } & A x \geq c \\
            & x \geq 0
        \end{align*}
        Prove that if $x^{*}$ satisfies $A x^{*}=c$ and $x^{*} \geq 0$, then $x^{*}$ is an optimal solution.
    \end{framed}
\newpage
    \begin{framed} 
    \item [\textbf{P. 7}] Consider the following pair of linear programming problems:
        \begin{align*}
            \text{minimize } & c^{\prime} x & \text{maximize } & p^{\prime} b \\
            \text{subject to } & A x \geq b & \text{subject to } & p^{\prime} A \leq c^{\prime} \\
            & x \geq 0, & & p \geq 0 .
        \end{align*}
        Suppose at least one of these two problems has a feasible solution. Prove that the set of feasible solutions to at least 
        one of the two problems is unbounded.
    \end{framed}
    \begin{proof}
        We are proving Clark's Lemma. \\
        \begin{itemize}
            \item [1.] If the primal set us unbounded, the dual can be either bounded or unbounded. 
            \item [2.] If the primal set is bounded, consider the problem: 
                \begin{align*}
                    \text{min } & -c^\prime x & \text{max } & p^\prime b \\
                    \text{s.t. } & Ax \geq b & \text{s.t.} & p^\prime A \leq -c^\prime \\
                                 &x \geq 0, & & p \geq 0. 
                \end{align*} 
                Say $w$ is a solution to this problem, so $w^\prime A \leq c^\prime$ and $w \geq 0$. Say $d$ is a solution 
                to the original dual problem and let $\theta \geq 0$. If we take $d + \theta w$ and show that this is a feasible 
                solution to the original dual problem, we get that the dual is unbounded. Indeed, $d + \theta w \geq 0$ as 
                $d, \theta, w \geq 0$ and $d + \theta w^\prime A \leq c^\prime$ since $(\theta w)A \leq -c^\prime 
                \Rightarrow (d + \theta w)^\prime A \leq c^\prime$.
        \end{itemize} 
    \end{proof}
\newpage
    \begin{framed} 
    \item [\textbf{P. 10}] Consider the problem: 
        \begin{align*}
            & \min \max _{i=1, \ldots, m}\left(a_{i}^{\prime} x-b_{i}\right) \\
            & \text { s.t. } x \in \mathbb{R}^{n} .
        \end{align*} Let $v$ be the optimal value of the above optimization problem, let $A$ be the $m \times n$ matrix whose 
        rows are $a_{1}^{\prime}, \ldots, a_{m}^{\prime}$, and let $b$ be the vector whose components are $b_{1}, \ldots, b_{m}$.
        \begin{itemize}
            \item [a.] Consider any vector $p \in \mathbb{R}^{m}$ such that $p^{\prime} A=0^{\prime}, p \geq 0$, and 
                $\sum_{i=1}^{m} p_{i}=1$. Show that $-p^{\prime} b \leq v$.
            \item [b.] In order to obtain the best possible lower bound of the form obtained in part (a), we consider the linear program:
                \begin{align*}
                    \max -p^{\prime} b \\
                    \text { s.t. } p^{\prime} A=0^{\prime} \\
                    \sum_{i=1}^{m} p_{i}=1 \\
                    p \geq 0.
                \end{align*} 
                Show that the optimal cost of the above linear program is $v$. 
        \end{itemize}
    \end{framed}
\end{jacklist}
\end{document}
